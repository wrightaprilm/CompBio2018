\documentclass[12pt]{article}
\usepackage[a4paper,margin=1cm,footskip=.5cm]{geometry}
\usepackage[]{authblk}
\usepackage{graphicx}
\usepackage{hanging}
\usepackage{indentfirst}
\usepackage{fancyhdr}
\usepackage{setspace}
\usepackage{enumitem}
\usepackage{multicol}
\usepackage{color}
\usepackage{lipsum}
\usepackage[breaklinks]{hyperref}
\usepackage[all]{hypcap}
\definecolor{darkblue}{rgb}{0.0,0.0,0.5}
%\onehalfspace
\setlength{\evensidemargin}{0in}
\setlength{\oddsidemargin}{0in}
\setlength{\paperheight}{11in}
\setlength{\paperwidth}{8.75in}
\setlength{\tabcolsep}{0in}
\setlength{\textheight}{9in} % This changed from 9in
\setlength{\textwidth}{6.5in}
\setlength{\topmargin}{0.15in}
\setlength{\topskip}{0mm}
\setlength{\voffset}{-0.5in}
\setlength{\parindent}{1cm}
%\setlength{\headheight}{99pt}
\setlength{\headheight}{0.5in}
\parskip = 1mm
\pagestyle{plain}

\definecolor{citescol}{RGB}{73,0,165}
\definecolor{urlscol}{RGB}{3,0,255}
\definecolor{linkscol}{RGB}{187,24,0}
\definecolor{isured}{RGB}{204,0,51}
\definecolor{isukhaki}{RGB}{76,69,43}
\definecolor{coolblack}{rgb}{0.0, 0.18, 0.39}
\let\oldtextbf\textbf
\renewcommand{\textbf}[1]{\textcolor{coolblack}{\oldtextbf{#1}}}

\hypersetup{colorlinks=true,linkcolor=linkscol,citecolor=citescol,urlcolor=urlscol}

\begin{document}

\thispagestyle{fancy}
\begin{flushright}
\today
\end{flushright}
\vspace{2mm}
\begin{flushleft}
\textbf{Course Title:} Computational Biology \\
\textbf{Course Number:} GBIO 408/508\\


\textbf{Course Date:} Fall 2018 \\

\textbf{Course Meeting Times:} Tuesday-Thursday 1-3:50 \\
\textbf{Course Meeting Location:} Biology Building 412 \\
\end{flushleft}

\bigskip

\begin{flushleft}
\textbf{Course Faculty:} Dr. April Wright \\
\textbf{Office:} Biology Building 403\\
\textbf{Office Phone:} 5556 \\
\textbf{Email:} april.wright@selu.edu   \\
\textbf{Office Hours:} Monday and Wednesday, 10:45-12:30, and by appointment \\

\end{flushleft}

\bigskip

\begin{flushleft}

\textbf{Course Description}
In this course, we will explore the fundamentals of managing data and performing analyses computationally. This course is intended for biologists who do not have experience with programming or computational sciences. 

\end{flushleft}

\bigskip
\begin{flushleft}

\textbf{Course Objectives}

\begin{itemize}

\item Work with data using programming
\item Make appropriate visualizations of data
\item Create computational reports from raw data
\item Use revision management to track changes to data and code
\item Distribute analyses to colleagues

\end{itemize}
\end{flushleft}

\bigskip

\begin{flushleft}
\textbf{Assessment}
A grade of `C" or better in this course is required to satisfy the curriculum requirements for the College of Science and Technology. There are a total of 700 points in this course. They are distributed as follows:

\begin{itemize}
\item \textbf{Projects:} 100 pts each
\item \textbf{Homeworks:} 100 pts (10 points each)
\item \textbf{Classroom exercises:} 100 pts
\item \textbf{Presentation:} 100 pts
\end{itemize}	

\bigskip

\textbf{Grades will be assigned as follows:}

A: 630-700 points, B: 560-629 pts, C: 490-559 pts, D: 420-489, F: Below 419 pts

\end{flushleft}

\bigskip

\begin{flushleft}
\textbf{Attendance and Make-Up}
\end{flushleft}

 Attendance is expected, and completion grade activities will be turned in almost every class period. Homeworks will be posted via the course Moodle. Homework will be due every Friday on non-exam weeks. Because they will be available for the entire week before they are due, \textbf{no make ups} will be available for assignments unless prior approval is granted. \par
 If you are aware in advance of absences, please let me know. The more information we have, the easier it is for me to accommodate you. \par

\begin{flushleft}
\textbf{Important Dates}
\end{flushleft}

\begin{itemize}
\item Sept. 12: Academic Checkpoint 1	
\item Oct. 10: Academic Checkpoint 2	
\item Nov 2: Withdrawal deadline
\item Nov 30: Last day of classes

\end{itemize}

\begin{flushleft}
\textbf{Schedule}
\end{flushleft}

Lectures will be posted the day before they are given by 5 pm.

\begin{itemize}
\item Week of Aug. 20: Introduction to Python and JupyterHub
\item Week of Aug. 27: Working with Data I
\item Week of Sept. 3: Working with Data II
\item Week of Sept. 10: Visualization, Project 1 due
\item Week of Sept 17: Project Structuring 
\item Week of Sept 24: Programming I
\item Week of Oct 1: Revision management
\item Week of Oct 8: Programming II
\item Week of Oct 14: Project II due
\item Week of Oct 21: Making a Python Package
\item Week of Oct 28: Deploying an analysis
\item Week of Nov 6: Making notebooks and posters
\item Week of Nov 12: Project III
\item Week of Nov 19: Advanced Topics
\item Week of Nov 25: Final project workweek
\item Final: Monday, December 3, 10:15 a.m. - 12:15 p.m. 

\end{itemize}
\end{document}
